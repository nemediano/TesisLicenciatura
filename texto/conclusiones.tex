\chapter*{Conclusiones}
\addcontentsline{toc}{chapter}{\numberline{}Conclusiones}
\markboth{CONCLUSIONES}{CONCLUSIONES}

Después de haber hecho pruebas y evaluado los resultados obtenidos presentó las siguientes conclusiones.

La más importante: las ecuaciones diferenciales, permiten modelar gráficamente el comportamiento físico de un cuerpo neumático que interactúa con un cuerpo rígido en tiempo real.
También hay que recalcar que este modelo tiene sustento en la física.
Es decir es un modelo basado en física y no en geometría.

Creo también que la técnica aquí usada; la de utilizar un gas ideal dentro del cuerpo flexible, es muy buena para enfrentar este tipo de problemas.
La técnica nos proporciona todo lo que desearíamos para hacer una animación: es computacionalmente barata; al menos lo suficiente para alcanzar tiempo real, y es físicamente adecuada para modelar el gas.

La implementación como se describe, depende poco del poder gráfico.
El render sigue siendo una operación relativamente barata \emph{en comparación} con los cálculos numéricos.
Dentro de los cálculos numéricos, lo que más contribuye es el método numérico de Runge Kutta seguido de la acumulación de la fuerza del gas.

Si bien el utilizar el método de Euler hace considerablemente más rápido de ejecutar el programa, recomiendo usar el método de Runge Kutta.
Esto ofrece una mayor estabilidad ante la variación de las constantes, lo que se traduce en la posibilidad de una mejor interacción.
También creo que este método no es tan complejo de implementar.

El \emph{\textenglish{damping}} es un parámetro que debe de estar presente.
Sin embargo, requiere tener mucho cuidado al determinar un valor adecuado.
Considero que el valor más adecuado es el más grande posible sin que la animación explote.
Este modelo es muy sensible a las variaciones (incluso \emph{pequeñas}) de éste parámetro.

Se recomienda ampliamente que al momento de programar se tome tiempo para hacer una adecuada elección de las estructuras de datos.
Guardar todo en una estructura de datos lineal simplifica bastante la programación.
De igual manera, recomiendo tener diferentes maneras de acceder a las propiedades de las partículas, ya sea por los resortes o por medio de las caras.

El modelo del gas ideal es un modelo muy recomendado para simular cuerpos flexibles.
Con una adecuada elección de los parámetros se puede tener un comportamiento bastante realista.
Por lo que considero que el objetivo fundamental de este trabajo se cumple.

Hay muchas mejoras posibles para esta implementación, pienso que hay campo para futuras investigaciones en los siguientes detalles:

\begin{itemize}
 \item Investigar sobre un método numérico más eficiente, aquel que dé más rapidez sin perder estabilidad.
 \item El manejo de colisiones más eficiente.
\end{itemize}

Para el método numérico se hicieron pruebas con el integrador de Verlet.
Sin embargo, abandoné ese camino porque en un esquema como ése no se toma en cuenta la velocidad de la partícula.
Por lo que no encontré manera de responder a las colisiones.
Creo que investigar la manera de incluir una fuerza de impulso como respuesta a la colisión y utilizar el integrador de Verlet daría resultado.

Hay también que considerar una respuesta a las colisiones que conlleven una pérdida de energía al momento de la colisión.
Es decir, eliminar el supuesto de colisiones perfectamente elásticas.
Aunque escribí en el segundo capítulo sobre ellas, no fueron implementadas en el programa.

Mi forma de manejar tanto la detección como la respuesta de las colisiones  no es la más eficiente, pues pruebo una a una las partículas del cuerpo flexible.
Alguna estructura de datos geométrica (como un \emph{\textenglish{octree}} o un \emph{\textenglish{range tree})} que probara sólo las partículas cercanas haría mejor la detección.

También hay que considerar que el cuerpo flexible puede colisionar consigo mismo.
Es decir, hace falta probar colisiones de las caras del cuerpo con cada una de las partículas y luego responderlas.
Considero que en cualquier configuración del cuerpo flexible que no sea un volumen convexo éste problema estará presente.
